\def\firstname{Tim}
\def\lastname{Dahmen}
\def\aufgabenblatt{4}
\include{../common/header.tex}

\begin{document}

\thispagestyle{page1} 

\section{Overview}

In this exercise sheet, you will extract features from images, which allows us to transfer image data (Pixels) to tabular data, accessible by conventional machine learning.

\subsection{Flood Fill Algorithm}

We start by implementing the flood fill algorithm, an algorithm used to mark connected areas in images. 

\begin{enumerate}

\item[a)] Describe your test strategy.

\item[b)] Implement unittests for the floodfill algorithm.

\item[c)] Implement the the floodfill algorithm.

\item[d)] Apply the floodfill algorithm to extract the particles in the file \texttt{shapes\_binary.png}. Results should be stored as pixel coordinates for each particle in a file called \texttt{particles.csv}.

\item[e)] Visualize the detected particles. Do all results seem to be legitimate particles? If not, what is the reason and what can be done about it?

\end{enumerate}

I needed the following time to complete the task:

\subsection{Isovalue Contours}

Next, we detect the particles using isocontours. 

\begin{enumerate}

\item[a)] Describe your test strategy and implement unittests for the isocontours algorithm.

\item[b)] Implement an algorithm that extracts isocontours from binary images. 

\item[c)] Apply the algorithm to extract the contours from the image in the file \texttt{shapes\_binary.png}. Results should be stored as pixel coordinates for each particle in a file called \texttt{contours.csv}.

\end{enumerate}

I needed the following time to complete the task:

\subsection{Particle Parameters}

For each of the particles detected before, we calculate several parameters:

\begin{enumerate}

\item[a)] Compute the area of particle using the flood fill algorithm

\item[b)] Compute the length of the contour.

\item[c)] Compute the convex hull of the contour using code from \texttt{scipy.spatial}, and compute the porosity of the particle.

\item[d)] Compute the the best fitting ellipse using code from \texttt{skimage.measure}.

\end{enumerate}

I needed the following time to complete the task:

\subsection{Particle Classification}

With all the preparation done, we can now use machine learning techniques to classify the particles. 

\begin{enumerate}

\item[a)] Have a look at the folder \texttt{training\_data}. The folder contains images containing shapes, and csv files containing particle locations and ground truth classifications.
The first step is to calculate the required parameters for the files in the training data, and match it with the ground truth classifications.

\item[b)] Construct a decision tree using your code from exercise sheet 2. Use the tree to classify all particles in the file \texttt{shapes\_binary.png}. Store the result in a file called \texttt{test\_decision\_tree.csv}.

\item[c)] Now use the neural network from exercise sheet 3 for the same task. Store the result in a csv file called \texttt{test\_neural.csv}. 

\item[d)] Compare the results from exercise b) and c). Draw a conclusion of the comparison.

\end{enumerate}

I needed the following time to complete the task:

\end{document}